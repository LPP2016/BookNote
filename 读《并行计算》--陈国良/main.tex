\documentclass{article}

\usepackage{ctex}
\usepackage{cite}
\usepackage{bm}
\usepackage{setspace}
\usepackage{graphicx}
%\usepackage{float}
\usepackage{stfloats}
\usepackage{placeins}
\usepackage{subfigure}
\usepackage{listings}
\usepackage{url}
\usepackage[linesnumbered,boxed,ruled,commentsnumbered]{algorithm2e}%%算法包,注意设置所需可选项
\usepackage{amsmath}
\newtheorem{definition}{定义}[section]
\newtheorem{theorem}{定理}[section]
\newtheorem{proof}{证明}[section]
\usepackage{setspace} 
\usepackage{diagbox}
\usepackage{multirow}
\usepackage[colorlinks,
linkcolor=blue,
anchorcolor=blue,
citecolor=blue]{hyperref}

\begin{document}
	\title{陈国良. 并行计算:结构·算法·编程[M]. 1999.}
	\author{屈彬}
	\maketitle
	
	\section{并行计算机系统及其结构模型}
		\subsection{并行计算(2019年7月2日)}
			\subsubsection{并行计算与计算科学}
				\begin{definition}[应用需求]
					应用需求分为三类,\textbf{计算密集}(Compute-Intensive)型应用,如大型科学工程计算与数值模拟;\textbf{数据密集}(Data-Intensive)型应用,如数字图书馆、数据仓库、数据挖掘和计算可视化等;\textbf{网络密集}(Network-Intensive)型应用,如协同工作、遥控和远程医疗诊断等。
				\end{definition}
				\begin{definition}[高性能计算和通信]
					High Performance Computing and Communication, HPCC。
				\end{definition}
				\begin{definition}[加速战略计算创新]
					Accelerated Strategic Computing Initiative, ASCI。
				\end{definition}
				\begin{definition}[美国能源部三大高性能计算实验室]
					Lawrence Livermore、Los Alamos、Sandia。
				\end{definition}
		\subsection{并行计算系统互连(2019年7月3日)}
			\subsubsection{系统互连}
				\begin{definition}[机群网络分类]
					\begin{enumerate}
						\item \textbf{系统域网络}:(System Area Network,SAN),连接短距离(3~25m)内的节点。
						\item \textbf{局域网络}:(Local Area Network,LAN),连接企事业单位内(500m~2km)内的节点。
						\item \textbf{节点内网络}:由\textbf{处理器总线}、\textbf{局部(本地)总线}、\textbf{存储器总线}构成。
						\item \textbf{小型机系统接口}:(Small Computer System Interface, SCSI)由\textbf{I/O总线}、\textbf{系统总线}构成。
					\end{enumerate}
				\end{definition}
				\begin{definition}[工作站机群]
					(Cluster of Workstations, COW),通过SAN/LAN互连的工作站组。
				\end{definition}
			\subsubsection{静态互连网络}
				\begin{definition}[对剖宽度]
					(Bisection Width):将网络划分为两等分所必须剪去的最少边数。
				\end{definition}
			\subsubsection{动态互连网络}
				\begin{definition}[总线分类]
					常用总线包括PCI、VME、Multibus、SBus、Microchannel和IEEE Futurebus。
				\end{definition}
				\begin{enumerate}
					\item \textbf{本地总线}:(Local Bus),CPU板级上的总线。
					\item \textbf{存储器总线}:存储器板级上的总线,主要指内存与CPU互连的总线。
					\item \textbf{数据总线}:I/O与通信板级上的总线,例如硬盘与内存之间的总线,以及网卡。
				\end{enumerate}
			\subsubsection{标准互连网络分类}
				\begin{enumerate}
					\item \textbf{光纤分布式数据接口}(Fiber Distributed Data Interface, FDDI),采用双向光纤令牌环提供100~200Mb/s的数据传输。
					\item \textbf{快速以太网}主流的互连网络,第三代以太网数据传输速度可达1Gb/s。
					\item \textbf{myrinet}由Myricom公司生产的千兆位包开关网。
					\item \textbf{高性能并行接口}(High Performance Parallel Interface, HiPPI),主要用于构筑异构计算机系统。
					\item \textbf{异步传输模式}(Asynchronous Transfer Mode, ATM),在光纤通信基础上建立起来的一种新的宽带综合业务数字网(B-ISDN)的交换技术。
				\end{enumerate}
\end{document}