\documentclass{article}

\usepackage{ctex}
\usepackage{cite}
\usepackage{bm}
\usepackage{setspace}
\usepackage{graphicx}
%\usepackage{float}
\usepackage{stfloats}
\usepackage{placeins}
\usepackage{subfigure}
\usepackage{listings}
\usepackage{url}
\usepackage[linesnumbered,boxed,ruled,commentsnumbered]{algorithm2e}%%算法包,注意设置所需可选项
\usepackage{amsmath}
	\newtheorem{definition}{定义}[section]
	\newtheorem{theorem}{定理}[section]
\usepackage{setspace} 
\usepackage{diagbox}
\usepackage{multirow}
\usepackage{multirow}
\usepackage[colorlinks,
linkcolor=blue,
anchorcolor=blue,
citecolor=blue]{hyperref}

\begin{document}
	\title{陈国良. 并行计算:结构·算法·编程[M]. 1999.}
	\author{屈彬}
	\maketitle
	
	\section{并行计算机系统及其结构模型}
		\subsection{并行计算(2019年7月2日)}
			\subsubsection{并行计算与计算科学}
				\begin{definition}[应用需求]
					应用需求分为三类,\textbf{计算密集}(Compute-Intensive)型应用,如大型科学工程计算与数值模拟;\textbf{数据密集}(Data-Intensive)型应用,如数字图书馆、数据仓库、数据挖掘和计算可视化等;\textbf{网络密集}(Network-Intensive)型应用,如协同工作、遥控和远程医疗诊断等。
				\end{definition}
				\begin{definition}[高性能计算和通信]
					High Performance Computing and Communication, HPCC。
				\end{definition}
				\begin{definition}[加速战略计算创新]
					Accelerated Strategic Computing Initiative, ASCI。
				\end{definition}
				\begin{definition}[美国能源部三大高性能计算实验室]
					Lawrence Livermore、Los Alamos、Sandia。
				\end{definition}
		\subsection{并行计算系统互连()}
\end{document}