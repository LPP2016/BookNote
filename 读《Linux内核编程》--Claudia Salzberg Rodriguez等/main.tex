\documentclass{article}

\usepackage{ctex}
\usepackage{cite}
\usepackage{bm}
\usepackage{setspace}
\usepackage{graphicx}
\usepackage{float}
\usepackage{stfloats}
\usepackage{placeins}
\usepackage{subfigure}
\usepackage{listings}
\usepackage{url}
\usepackage[linesnumbered,boxed,ruled,commentsnumbered]{algorithm2e}%%算法包,注意设置所需可选项
\usepackage{amsmath}
\newtheorem{definition}{定义}[section]
\newtheorem{theorem}{定理}[section]
\newtheorem{proof}{证明}[section]
\usepackage{setspace} 
\usepackage{diagbox}
\usepackage{multirow}
\usepackage[colorlinks,
linkcolor=blue,
anchorcolor=blue,
citecolor=blue]{hyperref}

\begin{document}
	\title{Rodriguez, Claudia Salzberg , and G. Fisch . Linux内核编程. 机械工业出版社, 2006. 陈莉君等 译}
	\author{屈彬}
	\maketitle
	
	这本书每一章背后都有习题,可根据习题的思路来阅读。
	\section{概述(2019年7月9日)}
		\subsection{Unix系统和Unix的克隆系统之间有何区别?}
			\textbf{MULTICS}(MULTiplexed Information and Computing Service):在当时支持\textbf{多道程序设计}操作系统的基础上开发的\textbf{多用户分时}系统,让每个用户都可以访问自己的终端。$\rightarrow$UNIX系统$\rightarrow$UNIX克隆系统:Berkeley UNIX(BSD)、AT\&T UNIX System III等$\rightarrow$Minix:一个教学用小型操作系统,最新版本为 MINIX 3,适合初学者学习$\rightarrow$Linux。
		\subsection{术语``基于Power的Linux''指什么?}
			\textbf{基于Power的Linux}指运行在Power/PowerPC处理器上的Linux系统。
			
			\textbf{Power/PowerPC}全称 Performance Optimization With Enhanced RISC - Performance Computing,是一种采用精简指令集(RISC)的处理器,由AIM联盟(苹果、IBM、摩托罗拉)研发。
		\subsection{什么是用户空间?什么是内核空间?}
			\textbf{用户空间}是用户、程序员开发或使用应用程序所在的逻辑层面,不能直接访问硬件资源,但可以通过内核定义的系统调用来访问。
			
			\textbf{内核空间}是硬件管理功能发挥作用的逻辑层面,对用户不可见。
		\subsection{用户空间的程序访问内核功能的接口是什么?}
			\textbf{系统调用}
		\subsection{用户UID与用户名有何联系?}
			UID是用户的唯一标识符,内核通过UID识别和验证用户的文件访问权限。\textbf{超级用户}/\textbf{root用户}的UID维0。不同用户的UID必然不同,而用户名可以相同。
		\subsection{列举文件与用户关联的方式}
			
\end{document}