\documentclass[a4paper,11pt,oneside]{article}

\usepackage{ctex}
\usepackage{cite}
\usepackage{bm}
\usepackage{setspace}
\usepackage{graphicx}
%\usepackage{float}
\usepackage{stfloats}
\usepackage{placeins}
\usepackage{subfigure}
\usepackage{listings}
\usepackage{url}
\usepackage[linesnumbered,boxed,ruled,commentsnumbered]{algorithm2e}%%算法包,注意设置所需可选项
\usepackage{amsmath}
	\newtheorem{definition}{定义}[section]
	\newtheorem{theorem}{定理}[section]
	\newtheorem{proof}{证明}[section]
	\newtheorem{inference}{推论}[section]
\usepackage{setspace} 
\usepackage{diagbox}
\usepackage{multirow}
\usepackage[colorlinks,
linkcolor=blue,
anchorcolor=blue,
citecolor=blue]{hyperref}

\begin{document}
	\title{Rockafellar, R. Tyrrell. Convex analysis. Vol. 28. Princeton university press, 1970. 盛宝怀 译}
	\author{屈彬}
	\maketitle
	
	\section{基本概念}
		\subsection{仿射集(2019年7月2日)}
			\begin{definition}[向量空间]
				$\textbf{R}$表示实数系,其中$\textbf{R}^{n}$表示由实$n$元组$\textbf{\textit{x}}=(\xi_{1},\xi_{2},...,\xi_{n})$所组成的向量空间。
			\end{definition}
			\begin{definition}[向量的内积]
				记两向量$\textbf{\textit{x}}$与$\textbf{\textit{x}}^{*}$在实数系$\textbf{R}^{n}$中的内积为
				$$<\textbf{\textit{x}},\textbf{\textit{x}}^{*}>=\xi_{1}\xi_{1}^{*}+\xi_{2}\xi_{2}^{*}+...+\xi_{n}\xi_{n}^{*}.$$
			\end{definition}
			\begin{definition}[线性变换]
				设向量空间$\textbf{R}^{n}$到向量空间$\textbf{R}^{m}$的线性变换为$\textbf{\textit{x}} \rightarrow \textbf{\textit{A}} \textbf{\textit{x}}$,其中$\textbf{\textit{A}}$为一$m\times n$阶实矩阵,称$\textbf{\textit{A}}$为线性变换$\textbf{R}^{n} \rightarrow \textbf{R}^{m}$的\textbf{变换矩阵}。记$\textbf{\textit{A}}^{*}$为$\textbf{\textit{A}}$的转置矩阵,$\textbf{\textit{y}}^{*}$为一实$m$元组,那么它们存在以下内积关系
				$$<\textbf{\textit{A}}\textbf{\textit{x}}, \textbf{\textit{y}}^{*}> = <\textbf{\textit{x}}, \textbf{\textit{A}}^{*} \textbf{\textit{y}}^{*}>.$$
			\end{definition}
			\begin{definition}[直线]
				设$\textbf{\textit{x}}$和$\textbf{\textit{y}}$为空间$\textbf{R}^{n}$中两个不同的点,记通过$\textbf{\textit{x}}$和$\textbf{\textit{y}}$的直线为$l(\textbf{\textit{x}},\textbf{\textit{y}})$,那么$l(\textbf{\textit{x}},\textbf{\textit{y}})$上所有的点都可以表示为$\textbf{\textit{x}}$坐标和$\textbf{\textit{y}}$坐标系数和为1的线性叠加,即
				$$l(\textbf{\textit{x}},\textbf{\textit{y}})\equiv(1-\lambda)\textbf{\textit{x}}+\lambda\textbf{\textit{y}}, \lambda \in \textbf{R}.$$
			\end{definition}
			\begin{definition}[仿射集]
				\label{def:5}
				令$M$为$\textbf{R}^{n}$中的一个子集,若$\forall \textbf{\textit{x}}, \textbf{\textit{y}} \in M \& \forall \lambda \in \textbf{R}$都有
				$$(1-\lambda)\textbf{\textit{x}}+\lambda\textbf{\textit{y}} \in M$$
				即经过子集$M$中任意两点的直线完全处于$M$内,则称$M$为$\textbf{R}^{n}$中的\textbf{仿射集}(affine set),又称作\textbf{仿射流形}(affine manifold)、\textbf{仿射变量}(affine variety)、\textbf{线形变量}(linear variety)或称$M$是\textbf{平坦的}(flat)。空集$\emptyset$和全集$\textbf{R}^{n}$一定是空间$\textbf{R}^{n}$中的仿射集。
			\end{definition}
			\begin{definition}[线性子空间]
				若$M$是$\textbf{R}^{n}$的一个线性子空间(简称子空间),那么以下条件必须同时成立
				\begin{enumerate}
					\item $\textbf{0} \in M$。
					\item $\forall \textbf{x} \in M \& \lambda \in \textbf{R}$满足$\lambda \textbf{x} \in M$(乘法封闭性)。
					\item $\forall \textbf{x},\textbf{y} \in M$满足$\textbf{x}+\textbf{y} \in M$(加法封闭性)。
				\end{enumerate}
			\end{definition}
			\begin{theorem}
				\label{the:theorem1}
				$\textbf{R}^{n}$中包含原点的仿射集是$\textbf{R}^{n}$的子空间。
			\end{theorem}
			\begin{proof}
				若$M$为$\textbf{R}^{n}$的一个包含原点的仿射集,则$\textbf{0} \in M$。\\
				下面证明$M$满足乘法封闭性。因为$M$是一个包含原点的仿射集,那么对于原点$\textbf{0}$和$M$中任一点$\textbf{x}$,有
				$$(1-\lambda)\textbf{0}+\lambda \textbf{x} \in M \Leftrightarrow \lambda \textbf{x} \in M, \lambda \in \textbf{R}$$
				即$M$满足乘法封闭性。\\
				\\
				下面证明$M$满足加法封闭性。设$\textbf{x}$和$\textbf{y}$是$M$中的任意两点,因为$M$是一个仿射集,所以满足
				$$\frac{1}{2} \textbf{x} + \frac{1}{2} \textbf{y} \in M \Leftrightarrow \frac{1}{2} (\textbf{x} + \textbf{y}) \in M$$
				又因为$M$满足乘法封闭性,所以
				$$\textbf{x} + \textbf{y} \in M$$
				即$M$满足加法封闭性。\\
				\\
				综上,$M$是$\textbf{R}^{n}$的一个子空间。
			\end{proof}
			\begin{definition}[仿射集的平行]
				仿射集$M$与$L$平行是指存在$\textbf{a}$使得$M=L+\textbf{a}$。易证平行具有传递性,若仿射集$L_{1}$和$L_{2}$皆平行于$M$,那么$L_{1}$平行于$L_{2}$。
			\end{definition}
			\begin{theorem}
				\label{the:theorem2}
				$\textbf{R}^{n}$上的每个非空仿射集$M$一定平行于唯一的子空间$L$,其中$L$由下式给出。
				$$L=M-M=\{\textbf{x}-\textbf{y}|\textbf{x},\textbf{y}\in M\}$$
			\end{theorem}
			\begin{proof}
				首先证明$M$不能同时与两个不同的子空间平行。设$L_{1}$和$L_{2}$为$\textbf{R}^{n}$的两个子空间,且$L_{1}$和$L_{2}$皆平行于$M$,则$L_{1}$与$L_{2}$互相平行,那么存在$\textbf{a}$使得
				$$L_{2}=L_{1}+\textbf{a}$$
				因为$L_{1}$是一个子空间,所以$\textbf{0} \in L_{1}$,则$\textbf{a} \in L_{2}$。设$\textbf{x}$是$L_{1}$中任意一点,则$\textbf{x}+\textbf{a} \in L_{2}$。根据加法封闭性,得$\textbf{x} \in L_{2}$,那么$L_{1} \subseteq L_{2}$。同理可证$L_{2} \subseteq L_{1}$。因为$L_{1} \subseteq L_{2}$且$L_{2} \subseteq L_{1}$,所以$L_{1} = L_{2}$,子空间唯一性得证。\\
				\\
				下面证明子空间$L=M-M$。设$\textbf{x}$是$M$中任意一点,那么仿射集$M-\textbf{x}$平行于$M$且$M-\textbf{x}$包含$\textbf{0}$。那么根据定理\ref{the:theorem1},$M-\textbf{x}$是$\textbf{R}^{n}$中的一个子空间,记作$L$,则$L=M-\textbf{x}$。因为$\textbf{x}$是$M$中的任意一点,无论选择什么样的$\textbf{x}$,都会有$L=M-\textbf{x}$,所以$L=M-M$成立。\\
				\\
				综上,定理\ref{the:theorem2}得证。
			\end{proof}
			\begin{definition}[仿射集的维数]
				非空仿射集的维数等于与其平行的子空间的维数。特别地,定义空集$\emptyset$的维数为-1。维数为0、1、2的仿射集分别被称为点、线和平面。空间$\textbf{R}^{n}$中的$n-1$维仿射集被称为\textbf{超平面}(比空间少一维)。
			\end{definition}
			\begin{definition}[正交]
				若向量$\textbf{x}$与向量$\textbf{y}$的内积为0,即$<\textbf{x},\textbf{y}>=0$,则称\textbf{x}与\textbf{y}正交,记作$\textbf{x} \perp \textbf{y}$。
			\end{definition}
			\begin{definition}[正交补(orthogonal complement)]
				给定$\textbf{R}^{n}$的子空间$L$,若对于一切$\textbf{y} \in L$,都有$\textbf{x} \perp \textbf{y}$,这些由$\textbf{x}$构成的集合称为$L$的正交补,记作$L^{\perp}$。易证,空间$\textbf{R}^{n}$中子空间$L$的维数与其正交补$L^{\perp}$的维数之和等于空间的维数,即
				$$dim(L)+dim(L^{\perp})=n.$$
				另外,子空间的正交补的正交补是其本身,即
				$$L=(L^{\perp})^{\perp}$$
				借助正交补,可通过一维子空间刻画超平面。设$\textbf{b}$是一个非零向量,那么集合$\{\textbf{x}|\textbf{x} \perp \textbf{b}\}$表示一个$n-1$维的子空间(也表示一个经过原点的超平面),该子空间经过平移变换可表示任何与之平行的超平面,形式如下:
				$$\{\textbf{x}|\textbf{x}+\textbf{b}\}+\textbf{a}=\{\textbf{x}+\textbf{a}<\textbf{x},\textbf{b}>\}.$$
			\end{definition}
			\begin{theorem}[超平面的表示]
				\label{the:3}
				给定$\beta \in \textbf{R}$以及非零$\textbf{b} \in \textbf{R}^{n}$,集合
				$$H=\{\textbf{x}|<\textbf{x},\textbf{b}>=\beta\}$$
				可表示$\textbf{R}^{n}$中的一个超平面。其中,向量$\textbf{b}$正交于超平面$H$,其余每个与$H$正交的向量为$\textbf{b}$的整倍数。
			\end{theorem}
			\begin{proof}
				略。
			\end{proof}
			\begin{theorem}[仿射集的表示]
				\label{the:4}
				空间$\textbf{R}^{n}$中的仿射集可通过一个$n$元线性方程组表示。给定$\textbf{b}\in \textbf{R}^{n}$及$m\times n$阶$(m \leq n)$实矩阵$\textbf{B}$,集合
				$$M=\{\textbf{x}\in \textbf{R}^{n}|\textbf{B}\textbf{x}=\textbf{b}\}$$
				为$\textbf{R}^{n}$中的一个仿射集。
			\end{theorem}
			\begin{proof}
				略。书中的证明不够严谨,应该仔细考虑$m>n$,即方程组为超定方程组的情形。
			\end{proof}
			\begin{theorem}[子空间的表示]
				\label{the:5}
				设$L$为$\textbf{R}^{n}$中的一个子空间,$\textbf{b}_{1},\textbf{b}_{2},...,\textbf{b}_{m}$为$L^{\perp}$的一组基,那么
				$$L=(L^{\perp})^{\perp}=\{\textbf{x}|\textbf{x}\perp\textbf{b}_{1},...,\textbf{x}\perp\textbf{b}_{m}\}=\{\textbf{x}|<\textbf{x},\textbf{b}>=0,i=1,...,m\}=\{\textbf{x}|\textbf{B}\textbf{x}=0\}$$
				其中,$\textbf{B}$为一个$m\times n$实矩阵,它的行向量为$\textbf{b}_{1},\textbf{b}_{2},...,\textbf{b}_{m}$。\\
				\\
				联系定理\ref{the:4},仿射集$M$可通过子空间$L$表示,存在$\textbf{a}\in \textbf{R}^{n}$使得
				$$M=L+\textbf{a}=\{\textbf{x}|\textbf{B}\textbf{x}=\textbf{b}\}$$
				其中,$\textbf{b}=\textbf{B}\textbf{a}$。
			\end{theorem}
			\begin{inference}
				$\textbf{R}_{n}$中的每个仿射集都是有限个超平面的交集。\\
				\\
				定理\ref{the:4}中仿射集的表示也可以写成
				$$M=\{\textbf{x}|\textbf{B}\textbf{x}=\textbf{b}\}=\{\textbf{x}|<\textbf{x},\textbf{b}_{1}>=\beta_{1},<\textbf{x},\textbf{b}_{2}>=\beta_{2},...,<\textbf{x},\textbf{b}_{m}>=\beta_{m}\}$$
				其中,$\textbf{b}_{i}$为$\textbf{B}$的第$i$个行(向量),$\beta_{i}$为$\textbf{b}$的第$i$个分量,而根据定理\ref{the:3}
				$$H_{i}=\{\textbf{x}|<\textbf{x},\textbf{b}_{i}>=\beta_{i}\}$$
				表示第$i$个超平面,则
				$$M=\bigcap_{i=1}^{m}H_{i}$$
				因为任意一个$m$维的仿射集也是$m+1$维空间的超平面,所以仿射集与仿射集的交集依然是仿射集。
			\end{inference}
			\begin{definition}[仿射包]
				\label{def:11}
				给定$S \subset \textbf{R}^{n}$($S$是一个集合),存在唯一一个包含$S$的最小的仿射集(例如,满足$S \subseteq M$的所有仿射集的交集)。这个仿射集被称为$S$的仿射包(afine hull),记作aff$S$。aff$S$可以由所有形式为$\lambda_{1}\textbf{x}_{1}+...+\lambda_{m}\textbf{x}_{m}$向量组成,其中$\textbf{x}_{i} \in S$且$\lambda_{1}+...+\lambda_{m}=1$。
			\end{definition}
			\begin{definition}[仿射无关]
				$m+1$个向量$\textbf{b}_{0},\textbf{b}_{1},...,\textbf{b}_{m}$的集合若使得aff$\{\textbf{b}_{0},\textbf{b}_{1},...,\textbf{b}_{m}\}$的维数为$m$,则称它们是仿射无关(affinely independent)的。反之,若aff$\{\textbf{b}_{0},\textbf{b}_{1},...,\textbf{b}_{m}\}$的维数小于$m$,则称它们是仿射相关的。\\
				\\
				其中,以$\textbf{b}_{0}$的模作为aff$\{\textbf{b}_{0},\textbf{b}_{1},...,\textbf{b}_{m}\}$到原点的距离,那么$\textbf{b}_{0}$满足
				$$aff\{\textbf{b}_{0},\textbf{b}_{1},...,\textbf{b}_{m}\}=L+\textbf{b}_{0}$$
				其中
				$$L=aff\{\textbf{0},\textbf{b}_{1}-\textbf{b}_{0},...,\textbf{b}_{m}-\textbf{b}_{0}\}$$
				是唯一平行于aff$\{\textbf{b}_{0},\textbf{b}_{1},...,\textbf{b}_{m}\}$的子空间。$L$的维数为$m$当且仅当向量$\textbf{b}_{1}-\textbf{b}_{0},...,\textbf{b}_{m}-\textbf{b}_{0}$是线性无关的。\\
				\\
				结合定理\ref{the:4}和定理\ref{the:5},若仿射集$M$中的向量$\textbf{b}_{0},\textbf{b}_{1},...,\textbf{b}_{m}$是仿射无关的,则$M$中的任一向量$\textbf{x}$具有以下表达形式
				$$\textbf{x}=\lambda_{1}(\textbf{b}_{1}-\textbf{b}_{0})+...+\lambda_{m}(\textbf{b}_{m}-\textbf{b}_{0})+\textbf{b}_{0} = \lambda_{1}\textbf{b}_{1}+...+\lambda_{m}\textbf{b}_{m}+\textbf{b}_{0}-\sum_{i=1}^{m}\lambda_{i}\textbf{b}_{0}$$
				即
				$$\textbf{x}=\sum_{i=0}^{m}\lambda_{i}\textbf{b}_{i}, \lambda_{0}=-\sum_{i=1}^{m}\lambda_{i} \Leftrightarrow \sum_{i=0}^{m}\lambda_{i}=0$$
				其中,称$\lambda_{0},\lambda_{1},...,\lambda_{m}$为$M$中的\textbf{重心坐标系统}(barycentric coordinate system)。
			\end{definition}
			\begin{definition}[仿射变换]
				\label{def:13}
				给定一个从$\textbf{R}^{n}$到$\textbf{R}^{m}$的变换$T:\textbf{x} \rightarrow T(\textbf{x})$,若变换前后的满足系数保持,即对于每个$\textbf{R}^{n}$中的任意两点$\textbf{x}$和$\textbf{y}$及$\lambda \in \textbf{R}$,都有
				$$T((1-\lambda)\textbf{x}+\lambda\textbf{y})=(1-\lambda)T(\textbf{x})+\lambda T(\textbf{y})$$
				则称$T$为仿射变换。\\
				\\
				仿射变换的形式形如$T(\textbf{x})=\textbf{A}\textbf{x}+\textbf{a}$,其中$\textbf{A}$为线性变换,$\textbf{a}\in \textbf{R}^{m}$且$\textbf{a}=T(\textbf{0})$。
			\end{definition}
			\begin{theorem}
				\label{the:1.6}
				如果一个从$\textbf{R}^{n}$到$\textbf{R}^{m}$的映射$T$为仿射变换,则对于$\textbf{R}^{n}$中的每个仿射集$M$,像集$T(M)=\{T(\textbf{x})|\textbf{x}\in M\}$也为$\textbf{R}^{m}$中的仿射集。特别地,仿射变换保持仿射包不变:
				$$aff(T(S))=T(affS)$$
			\end{theorem}
			\begin{proof}
				首先证明像集$T(M)$也为$\textbf{R}^{m}$中的仿射集。设$\textbf{x}$和$\textbf{y}$分别为$\textbf{R}^{n}$中仿射集$M$上的任意两点,根据定义\ref{def:5},对于任意$\lambda \in \textbf{R}$有
				$$(1-\lambda)\textbf{x}+\textbf{y}\in M$$
				根据定义\ref{def:13},经过仿射变换后,满足
				$$(1-\lambda)T(\textbf{x})+\lambda T(\textbf{y})=T((1-\lambda)\textbf{x}+\lambda \textbf{y})\in T(M)$$
				因此像集$T(M)$也为$\textbf{R}^{m}$中的仿射集。\\
				\\
				下面证明仿射包保持不变。设$S$是$\textbf{R}^{n}$中的一个集合,aff$S$为$S$的仿射包,根据定义\ref{def:11},aff$S$可以由所有形式为$\lambda_{1}\textbf{x}_{1}+...+\lambda_{m}\textbf{x}_{m}$的向量组成,其中$\textbf{x}_{i}\in S$且$\sum_{i=1}^{m}\lambda_{i}=1$。$T(S)$的仿射包$aff(T(S))$可以表示为$\lambda_{1}T(\textbf{x}_{1})+...+\lambda_{m}T(\textbf{x}_{m})$的形式。根据定义\ref{def:13},有
				$$\lambda_{1}T(\textbf{x}_{1})+...+\lambda_{m}T(\textbf{x}_{m})=T(\lambda_{1}\textbf{x}_{1}+...+\lambda_{m}\textbf{x}_{m})$$
				即
				$$aff(T(S))=T(aff(S))$$
				综上所述,定理\ref{the:1.6}得证
			\end{proof}
			\begin{theorem}
				\label{the:1.7}
				设$\{\textbf{b}_{0},\textbf{b}_{1},...,\textbf{b}_{m}\}$和$\{\textbf{b}_{0}^{'},\textbf{b}_{1}^{'},...,\textbf{b}_{m}^{'}\}$在$\textbf{R}^{n}$中是仿射无关的,则存在将$\textbf{R}^{n}$映射到自身的一一对应的仿射变换$T$使得对于$i=0,...,m$有$T(\textbf{b}_{i})=\textbf{b}_{i}^{'}$。如果$m=n$,那么$T$是唯一的。
			\end{theorem}
			\begin{proof}
				由于时间有限,若定理与多面体编译关系不大,则不再给出证明过程。
			\end{proof}
			进一步推广定理\ref{the:1.7},还可得到以下推论
			\begin{inference}
				设$M_{1}$和$M_{2}$为$\textbf{R}^{n}$中两个维数相同的仿射集,则存在$\textbf{R}^{n}$自身到自身的一一对应的仿射变换$T$,使得$T(M_{1})=M_{2}$
			\end{inference}
\end{document}