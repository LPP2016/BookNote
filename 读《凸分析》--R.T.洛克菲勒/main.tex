\documentclass[a4paper,11pt,oneside]{article}

\usepackage{ctex}
\usepackage{cite}
\usepackage{bm}
\usepackage{setspace}
\usepackage{graphicx}
%\usepackage{float}
\usepackage{stfloats}
\usepackage{placeins}
\usepackage{subfigure}
\usepackage{listings}
\usepackage{url}
\usepackage[linesnumbered,boxed,ruled,commentsnumbered]{algorithm2e}%%算法包,注意设置所需可选项
\usepackage{amsmath}
	\newtheorem{definition}{定义}[section]
	\newtheorem{theorem}{定理}[section]
	\newtheorem{proof}{证明}[section]
\usepackage{setspace} 
\usepackage{diagbox}
\usepackage{multirow}
\usepackage[colorlinks,
linkcolor=blue,
anchorcolor=blue,
citecolor=blue]{hyperref}

\begin{document}
	\title{Rockafellar, R. Tyrrell. Convex analysis. Vol. 28. Princeton university press, 1970. 盛宝怀 译}
	\author{屈彬}
	\maketitle
	
	\section{基本概念}
		\subsection{仿射集(2019年7月2日)}
			\begin{definition}[向量空间]
				$\textbf{R}$表示实数系,其中$\textbf{R}^{n}$表示由实$n$元组$\textbf{\textit{x}}=(\xi_{1},\xi_{2},...,\xi_{n})$所组成的向量空间。
			\end{definition}
			\begin{definition}[向量的内积]
				记两向量$\textbf{\textit{x}}$与$\textbf{\textit{x}}^{*}$在实数系$\textbf{R}^{n}$中的内积为
				$$<\textbf{\textit{x}},\textbf{\textit{x}}^{*}>=\xi_{1}\xi_{1}^{*}+\xi_{2}\xi_{2}^{*}+...+\xi_{n}\xi_{n}^{*}.$$
			\end{definition}
			\begin{definition}[线性变换]
				设向量空间$\textbf{R}^{n}$到向量空间$\textbf{R}^{m}$的线性变换为$\textbf{\textit{x}} \rightarrow \textbf{\textit{A}} \textbf{\textit{x}}$,其中$\textbf{\textit{A}}$为一$m\times n$阶实矩阵,称$\textbf{\textit{A}}$为线性变换$\textbf{R}^{n} \rightarrow \textbf{R}^{m}$的\textbf{变换矩阵}。记$\textbf{\textit{A}}^{*}$为$\textbf{\textit{A}}$的转置矩阵,$\textbf{\textit{y}}^{*}$为一实$m$元组,那么它们存在以下内积关系
				$$<\textbf{\textit{A}}\textbf{\textit{x}}, \textbf{\textit{y}}^{*}> = <\textbf{\textit{x}}, \textbf{\textit{A}}^{*} \textbf{\textit{y}}^{*}>.$$
			\end{definition}
			\begin{definition}[直线]
				设$\textbf{\textit{x}}$和$\textbf{\textit{y}}$为空间$\textbf{R}^{n}$中两个不同的点,记通过$\textbf{\textit{x}}$和$\textbf{\textit{y}}$的直线为$l(\textbf{\textit{x}},\textbf{\textit{y}})$,那么$l(\textbf{\textit{x}},\textbf{\textit{y}})$上所有的点都可以表示为$\textbf{\textit{x}}$坐标和$\textbf{\textit{y}}$坐标系数和为1的线性叠加,即
				$$l(\textbf{\textit{x}},\textbf{\textit{y}})\equiv(1-\lambda)\textbf{\textit{x}}+\lambda\textbf{\textit{y}}, \lambda \in \textbf{R}.$$
			\end{definition}
			\begin{definition}[仿射集]
				令$M$为$\textbf{R}^{n}$中的一个子集,若$\forall \textbf{\textit{x}}, \textbf{\textit{y}} \in M \& \forall \lambda \in \textbf{R}$都有
				$$(1-\lambda)\textbf{\textit{x}}+\lambda\textbf{\textit{y}} \in M$$
				即经过子集$M$中任意两点的直线完全处于$M$内,则称$M$为$\textbf{R}^{n}$中的\textbf{仿射集}(affine set),又称作\textbf{仿射流形}(affine manifold)、\textbf{仿射变量}(affine variety)、\textbf{线形变量}(linear variety)或称$M$是\textbf{平坦的}(flat)。空集$\emptyset$和全集$\textbf{R}^{n}$一定是空间$\textbf{R}^{n}$中的仿射集。
			\end{definition}
			\begin{definition}[线性子空间]
				若$M$是$\textbf{R}^{n}$的一个线性子空间(简称子空间),那么以下条件必须同时成立
				\begin{enumerate}
					\item $\textbf{0} \in M$。
					\item $\forall \textbf{x} \in M \& \lambda \in \textbf{R}$满足$\lambda \textbf{x} \in M$(乘法封闭性)。
					\item $\forall \textbf{x},\textbf{y} \in M$满足$\textbf{x}+\textbf{y} \in M$(加法封闭性)。
				\end{enumerate}
			\end{definition}
			\begin{theorem}
				\label{the:theorem1}
				$\textbf{R}^{n}$中包含原点的仿射集是$\textbf{R}^{n}$的子空间。
			\end{theorem}
			\begin{proof}
				若$M$为$\textbf{R}^{n}$的一个包含原点的仿射集,则$\textbf{0} \in M$。\\
				下面证明$M$满足乘法封闭性。因为$M$是一个包含原点的仿射集,那么对于原点$\textbf{0}$和$M$中任一点$\textbf{x}$,有
				$$(1-\lambda)\textbf{0}+\lambda \textbf{x} \in M \Leftrightarrow \lambda \textbf{x} \in M, \lambda \in \textbf{R}$$
				即$M$满足乘法封闭性。\\
				下面证明$M$满足加法封闭性。设$\textbf{x}$和$\textbf{y}$是$M$中的任意两点,因为$M$是一个仿射集,所以满足
				$$\frac{1}{2} \textbf{x} + \frac{1}{2} \textbf{y} \in M \Leftrightarrow \frac{1}{2} (\textbf{x} + \textbf{y}) \in M$$
				又因为$M$满足乘法封闭性,所以
				$$\textbf{x} + \textbf{y} \in M$$
				即$M$满足加法封闭性。\\
				综上,$M$是$\textbf{R}^{n}$的一个子空间。
			\end{proof}
			\begin{definition}[仿射集的平行]
				仿射集$M$与$L$平行是指存在$\textbf{a}$使得$M=L+\textbf{a}$。易证平行具有传递性,若仿射集$L_{1}$和$L_{2}$皆平行于$M$,那么$L_{1}$平行于$L_{2}$。
			\end{definition}
			\begin{theorem}
				\label{the:theorem2}
				$\textbf{R}^{n}$上的每个非空仿射集$M$一定平行于唯一的子空间$L$,其中$L$由下式给出。
				$$L=M-M=\{\textbf{x}-\textbf{y}|\textbf{x},\textbf{y}\in M\}$$
			\end{theorem}
			\begin{proof}
				首先证明$M$不能同时与两个不同的子空间平行。设$L_{1}$和$L_{2}$为$\textbf{R}^{n}$的两个子空间,且$L_{1}$和$L_{2}$皆平行于$M$,则$L_{1}$与$L_{2}$互相平行,那么存在$\textbf{a}$使得
				$$L_{2}=L_{1}+\textbf{a}$$
				因为$L_{1}$是一个子空间,所以$\textbf{0} \in L_{1}$,则$\textbf{a} \in L_{2}$。设$\textbf{x}$是$L_{1}$中任意一点,则$\textbf{x}+\textbf{a} \in L_{2}$。根据加法封闭性,得$\textbf{x} \in L_{2}$,那么$L_{1} \subseteq L_{2}$。同理可证$L_{2} \subseteq L_{1}$。因为$L_{1} \subseteq L_{2}$且$L_{2} \subseteq L_{1}$,所以$L_{1} = L_{2}$,子空间唯一性得证。\\
				下面证明子空间$L=M-M$。设$\textbf{x}$是$M$中任意一点,那么仿射集$M-\textbf{x}$平行于$M$且$M-\textbf{x}$包含$\textbf{0}$。那么根据定理\ref{the:theorem1},$M-\textbf{x}$是$\textbf{R}^{n}$中的一个子空间,记作$L$,则$L=M-\textbf{x}$。因为$\textbf{x}$是$M$中的任意一点,无论选择什么样的$\textbf{x}$,都会有$L=M-\textbf{x}$,所以$L=M-M$成立。\\
				综上,定理\ref{the:theorem2}得证。
			\end{proof}
		
\end{document}